\chapter{Semantica operazionale ``big step''}

La semantica operazionale "big step" è un particolare tipo di semantica operazionale che descrive formalmente come il risultato complessivo dell'esecuzione di un'operazione è ottenuto. Per fare questo, sfrutta schemi di assiomi e regole che hanno la seguente forma:

$$
\prooftree
	premessa_1
	\cdots
	premessa_k
   \justifies
   		conclusioni
	\using
		(nome \; regola)
\endprooftree
$$

\subsection{Un primo esempio di applicazione}
Consideriamo un linguaggio con la seguente sintassi:
$$ Exp \ni E ::= n \bigm| (E+E) \bigm| \cdots $$
dove $n$ si espande nei naturali $1, 2, \dots$ 
I punti di sospensione garantiscono la possibilità di inserire nuove espressioni.
Allora una sua possibile semantica big step potrebbe essere, ad esempio:

$$
\prooftree
   \justifies
   		n \Downarrow n
	\using
		(B-NUM)
\endprooftree
$$

$$
\prooftree
	E_1 \Downarrow n_1 \; \; E_2 \Downarrow n_2
   \justifies
   		(E_1 + E_2) \Downarrow n_3
	\using
		(B-ADD)
\endprooftree
$$