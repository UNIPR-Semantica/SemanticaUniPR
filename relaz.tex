\chapter{Relazioni tra le semantiche}
In questo capitolo verranno introdotti alcuni esercizi e alcune osservazioni
che non riguardano una specifica semantica vista nei capitoli precedenti,
ma le affrontano un po' tutte.

\subsection{Esercizio 1} \marginpar{Mancino}
Aggiungere al linguaggio WHILE visto a lezione l'espressione:
\[ Exp \ni E ::= \dots \vbar \textrm{$B$ ? $E_2$ : $E_3$} \]
definendone la semantica operazionale Big Step, Small Step e denotazionale.

\begin{proof}[Svolgimento]
Di seguito illustriamo le regole e gli assiomi in semantica Small Step:
\[
\prooftree
  \langle B,s \rangle \textrm{ $\rightarrow$ } \langle B', s' \rangle
  \justifies
    \langle \textrm{$B$ ? $E_2$ : $E_3$, $s$ $\rangle$
    $\rightarrow$ $\langle$ $B'$ ? $E_2$ : $E_3$, $s'$ $\rangle$}
  \thickness=0.08em
\endprooftree 
\]

\[
\prooftree
  \justifies
    \textrm{ $\langle$ tt ? $E_2$ : $E_3$, $s$
    $\rangle$ $\rightarrow$ $\langle$ $E_2$, $s'$ $\rangle$}
  \thickness=0.08em
\endprooftree
\]

\[ 
\prooftree
  \justifies
    \textrm{$\langle$ ff ? $E_2$ : $E_3$, $s$
    $\rangle$ $\rightarrow$ $\langle$ $E_3$, $s'$ $\rangle$}
  \thickness=0.08em
\endprooftree
\]

Di seguito illustriamo le regole e gli assiomi in semantica Big Step:
\[
\prooftree
  \langle B,s \rangle \Downarrow \langle \textrm{tt}, s' \rangle \quad
  \langle E_2,s' \rangle \Downarrow \langle n, s'' \rangle
  \justifies
     \langle \textrm{$B$ ? $E_2$ : $E_3$, $s$ $\rangle$
     $\Downarrow \langle n$, $s'' \rangle$}
  \thickness=0.08em
\endprooftree
\]

\[
\prooftree
  \langle B,s \rangle \Downarrow \langle \textrm{ff}, s' \rangle \quad
  \langle E_3,s' \rangle \Downarrow \langle n, s'' \rangle
  \justifies
     \langle \textrm{$B$ ? $E_2$ : $E_3$, $s \rangle \Downarrow \langle n, s'' \rangle$}
  \thickness=0.08em
\endprooftree
\]

Di seguito illustriamo la definizione dell'espressione in semantica denotazionale:
\[
\llbracket \textrm{$B$ ? $E_2$ : $E_3 \rrbracket (s) $} =
\begin{cases}
    \varepsilon \llbracket E_2 \rrbracket (s)
    & \textrm{se } \emph{B} \llbracket B \rrbracket (s) = \textrm{tt} \\
    \varepsilon \llbracket E_3 \rrbracket (s)
    & \textrm{se } \emph{B} \llbracket B \rrbracket (s) = \textrm{ff} \\
    \bot & \textrm{altrimenti}
\end{cases}
\]
\end{proof}
